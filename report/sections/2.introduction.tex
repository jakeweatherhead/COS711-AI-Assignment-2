\section{INTRODUCTION}

The given dataset contains 2,803 almond instances in CSV format, each described by length, width, and other size-related metrics recorded from a 
previous collection of almond images. The task was to build a neural network that could accurately classify almond 
instances into one of three types—Mamra, Sonora, and Regular. This required iterative experimentation and fine-tuning to minimize loss and overfitting 
and maximize accuracy on unseen test cases.

After reviewing the findings in \cite{sun2017revisitingunreasonableeffectivenessdata}, which detail the effectiveness of large-scale datasets 
for training neural networks, it was decided to augment the original dataset with 20,000 additional rows of synthetic data. As will be shown, 
this augmentation resulted in significant improvements in the classifier's performance when tested 
on unseen almonds from the original dataset of real-world observations.

Initially, the characteristics of the original dataset were analyzed and it was found that the three almond types are evenly represented, 
with virtually no skew toward any particular almond-type. Then, the per-feature value distributions were visualized to establish a baseline to adhere to
when imputing missing values. Additionally, the number of missing values and outliers per feature was recorded.

The chosen method of mitigating outliers was the $Z$-score method, which was found to have the greatest benefit on performance amongst the
options for handling outliers. Missing values 
were handled using K-nearest neighbours imputation, which proved effective in providing the model with more complete training information while 
also preserving the underlying distribution of the affected feature.

After preprocessing the data, an optimal architecture was selected via cross-validation and the hyperparameters were fine-tuned to maximize 
performance on the validation set.

At the core of this report, a hybrid learning implementation was developed that allowed the averaging of two or more optimization algorithms used in tandem. 
By averaging their outputs, updates to the network proved more effective than when any single optimization algorithm was used on its own. 
As will be discussed in a subsequent section, using Adam and RProp in this hybrid implementation maximized the model's performance on unseen almond cases.

A grid search was performed across various learning rates and epoch counts to gain insight into configurations that might yield optimal 
performance on unseen data. 

A final round of manual fine-tuning led to the best observed performance accuracy on unseen test data, reported on in a later section.

% \subsection{Add subsections with this}